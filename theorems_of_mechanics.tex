%%%%%%%%%%%%%%%%%%%%
%
% Document created on May 2nd, 2020
% by Cristiano Garbellini
%
%%%%%%%%%%%%%%%%%%%%

\documentclass[a4paper,11pt]{article}
\usepackage[T1]{fontenc}
\usepackage[utf8]{inputenc}
\usepackage[french,italian,english]{babel}
\usepackage{amsmath}
\usepackage{amssymb}
\usepackage{amsthm}
\theoremstyle{plain}
\newtheorem{theorem}{Theorem}
\usepackage[colorlinks]{hyperref}

\author{Cristiano Garbellini}
\title{Theorems of Mechanics}

\begin{document}

\maketitle

\begin{theorem}[Leibnitz]
Si plusieurs forces, concourantes en un point et en équilibre, sont representées par des
lignes prises sur leurs directions et proportionnelles à leur intensités, et qu'aux extrémités
de ces lignes soint placés les centres de gravités de masses égales entre elles, le point de
concours des forces en équilibre sera le centre de gravité commun du système de ces
masses.
\end{theorem}


\end{document}

