%%%%%%%%%%%%%%%%%%%%
%
% Document created on May 2nd, 2020
% by Cristiano Garbellini
%
%%%%%%%%%%%%%%%%%%%%

\documentclass[a4paper,11pt]{article}
\usepackage[T1]{fontenc}
\usepackage[utf8]{inputenc}
\usepackage[french,italian,english]{babel}
\usepackage{amsmath}
\usepackage{amssymb}
\usepackage{amsthm}
\theoremstyle{plain}
\newtheorem{theorem}{Theorem}
\usepackage[colorlinks]{hyperref}

\author{Cristiano Garbellini}
\title{Theorems of Mechanics}

\begin{document}

\maketitle

\begin{theorem}[du Levier]	
Si un levier droit est chargé de deux poids quelconques placés de part et d'autre du point
d'appui, à des distances de ce point réciproquement proportionnelles aux même poids,
ce levier sera un équilibre, et son appui sera chargé de la somme des deux poids.
\end{theorem}

\begin{theorem}[des moments]
Deux forces, appliquées à des points quelconques d'un plan retenu par un point fixe, et
dirigées comme on voudra dans ce plan, sont en équilibre lorsqu'elles sont entre elles
en raison inverse des perpendiculaires abaissées de ce point sur leurs directions; car
on peut regarder ces perpendiculaires comme formant un levier angulaire dont le point
d'appui est le point fixe du plan.
\end{theorem}

\begin{theorem}[de Stevin]
L'équilibre de trois forces qui agissent sur un même point a lieu lorsque les forces sont
parallèles et proportionnelles aux trois côtés d'un triangle rectiligne quelconque.
\end{theorem}

\begin{theorem}[Composition des forces]
Deux forces quelconques qui agissent ensemble sur un même corps, sont équivalentes à
une seule représentée dans sa quantité et sa direction, par la diagonale du parallelogramme
dont les côtés représentent en particulier les quantités et les directions des deux forces
données.
\end{theorem}

\begin{theorem}[de Leibnitz]
Si plusieurs forces, concourantes en un point et en équilibre, sont representées par des
lignes prises sur leurs directions et proportionnelles à leur intensités, et qu'aux extrémités
de ces lignes soint placés les centres de gravités de masses égales entre elles, le point de
concours des forces en équilibre sera le centre de gravité commun du système de ces
masses.
\end{theorem}


\end{document}

